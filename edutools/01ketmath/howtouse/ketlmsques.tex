%%% タイトル ketlmsques
\documentclass[landscape,10pt]{ujarticle}
\special{papersize=\the\paperwidth,\the\paperheight}
\usepackage{ketpic,ketlayer}
\usepackage{ketslide}
\usepackage{amsmath,amssymb}
\usepackage{bm,enumerate}
\usepackage[dvipdfmx]{graphicx}
\usepackage{color}
\definecolor{slidecolora}{cmyk}{0.98,0.13,0,0.43}
\definecolor{slidecolorb}{cmyk}{0.2,0,0,0}
\definecolor{slidecolorc}{cmyk}{0.2,0,0,0}
\definecolor{slidecolord}{cmyk}{0.2,0,0,0}
\definecolor{slidecolore}{cmyk}{0,0,0,0.5}
\definecolor{slidecolorf}{cmyk}{0,0,0,0.5}
\definecolor{slidecolori}{cmyk}{0.98,0.13,0,0.43}
\def\setthin#1{\def\thin{#1}}
\setthin{0}
\newcommand{\slidepage}[1][s]{%
\setcounter{ketpicctra}{18}%
\if#1m \setcounter{ketpicctra}{1}\fi
\hypersetup{linkcolor=black}%

\begin{layer}{118}{0}
\putnotee{122}{-\theketpicctra.05}{\small\thepage/\pageref{pageend}}
\end{layer}\hypersetup{linkcolor=blue}

}
\usepackage{pict2e}
\usepackage[dvipdfmx,colorlinks=true,linkcolor=blue,filecolor=blue]{hyperref}

\setmargin{25}{145}{15}{100}

\ketslideinit

\pagestyle{empty}

\begin{document}

\begin{layer}{120}{0}
\putnotese{0}{0}{{\Large\bf
\color[cmyk]{1,1,0,0}

\begin{layer}{120}{0}
{\Huge \putnotes{60}{20}{KeTMath使い方}}
\putnotes{60}{50}{高遠節夫}
\putnotes{60}{60}{\ketcindy センター}
\putnotes{60}{70}{2021.10.27}
\end{layer}

}
}
\end{layer}

\def\mainslidetitley{22}
\def\ketcletter{slidecolora}
\def\ketcbox{slidecolorb}
\def\ketdbox{slidecolorc}
\def\ketcframe{slidecolord}
\def\ketcshadow{slidecolore}
\def\ketdshadow{slidecolorf}
\def\slidetitlex{6}
\def\slidetitlesize{1.3}
\def\mketcletter{slidecolori}
\def\mketcbox{yellow}
\def\mketdbox{yellow}
\def\mketcframe{yellow}
\def\mslidetitlex{62}
\def\mslidetitlesize{2}

\color{black}
\normalsize\bf\boldmath
\addtocounter{page}{-1}

\def\rad{\;\mathrm{rad}}
\newcommand{\hako}[2][6mm]{\fbox{$\mathstrut$\Ctab{#1}{#2}}}
\newcommand{\dint}{\displaystyle\int}
\newcommand{\dlim}{\displaystyle\lim}
\newcommand{\dc}{: \hspace{-1.4mm}:}
\setmargin{25}{150}{15}{100}%
%%%%%%%%%%%%%

%%%%%%%%%%%%%%%%%%%%

\newslide{課題ファイル(question)}

\vspace*{18mm}

%%repeat=3
\slidepage
\begin{itemize}
\item
ファイル名\\
\hspace*{2zw}question識別番号$-$最初の問題番号.txt
\item
識別番号 例えば 月日
\item
指定した問題番号をつける
\item
複数の問題を入れることができる\\
\hspace*{2zw}問題の間には空白行を入れる\\
\hspace*{2zw}問題番号は最初から1ずつ増やされる
\item
[例] question0615-1.txt,  question0615-3.txt\\
 (question0615-1には2つの問題が含まれる)
\end{itemize}
%%%%%%%%%%%%%

%%%%%%%%%%%%%%%%%%%%

\newslide{課題ファイルの書式}

\vspace*{18mm}

%%repeat=3
\slidepage
\begin{enumerate}[1.]
\item
Q    {\color{red}番号はファイル名から自動的に追加}\vspace{-3mm}
\item
 (問題文) {\color{red} 2行になるときは // で区切る}\\
$[1]$ (小問) {\color{red}ないときも [] とする}\\
$[2]$   (小問の数だけ)\vspace{-3mm}
\item
Sheet (解答欄)\\
$[1]$  {\color{red}\dc の後に配点(とMaxima不採点のとき \dc$-1$を追加)}\\
$[2]$\vspace{-3mm}
\item
Ans (正解)\\
$[1]$\\
$[2]$\vspace{-3mm}
\item
空白行をおいて次の問題を書く
\end{enumerate}
%%%%%%%%%%%%%

%%%%%%%%%%%%%%%%%%%%

\newslide{課題の作成例}

\vspace*{18mm}

%%repeat=3
\slidepage
\begin{itemize}
\item
[]Q\vspace{-2mm}
\item
[]問いに答えよ\vspace{-2mm}
\item
[]$[1]$sin(2x)の導関数の定義式をかけ\vspace{-2mm}
\item
[]$[2]$sin(2x)の0からfr(pi,2)までの定積分を求めよ\vspace{-2mm}
\item
[]Sheet\vspace{-2mm}
\item
[]$[1]$ =\ \   \dc 5\dc -1\vspace{-2mm}
\item
[]$[2]$ int(sin(2x),x,0,fr(pi,2))= \ \  \dc 5\vspace{-2mm}
\item
[]Ans\vspace{-2mm}
\item
[]$[1]$ lim(z,x,fr(sin(z)-sin(x),z-x))\vspace{-2mm}
\item
[]$[2]$ 1
\end{itemize}
%%%%%%%%%%%%%

%%%%%%%%%%%%%%%%%%%%

\newslide{Maximaによる正解の出力}

\vspace*{18mm}

%%repeat=3
\slidepage
\begin{itemize}
\item
Ansの直後に次のように書く\\
Mxcalc\\
$[1]$ sin(2x)\hfill{\color{red}i1に右辺をMaximaに変換した式を代入}\\
$[2]$ fr(pi,2)\hfill{\color{red}i2に右辺をMaximaに変換した式を代入}\\
f(x):i1\hfill{\color{red}関数定義やassume, declareなども入れられる}\\
o1:diff(f(x),x)\hfill{\color{red}o1に右辺(Maxima数式)の結果を代入}\\
o2:integrate(f(x),x,0,i2)\\
return o1\dc o2\hfill{\color{red}o1とo2の値を返す}
\item
toolketmath.cdyで「正解を出力」を実行する\\
  $\Longrightarrow$ 入力窓に結果が入る\\
  \phantom{$\Longrightarrow$} 同時にdataにmxans(qn).txtというファイルができる
\item
出力結果は \verb|[2*cos(2*x),1]|
\end{itemize}
\label{pageend}\mbox{}

\end{document}
